\section{Extensions to more expressive query languages}\label{sec:extensions}

\subsection{Unions of conjunctive queries}
In this section, we address the problem of finding a query plan for a UCQ.
Our approach is to transform a UCQ into an appropriate CQ to which the technique presented can be applied.
The obtained CQ can then be used to determine the relevant relations for the original UCQ.
We assume in the following, w.l.o.g., that the initial UCQ is constant-free, since any query containing constants can be transformed by  algorithm \ref{fig:algo-constants} into an equivalent query without constants.

Let $q$ be a constant-free UCQ over a schema $\R$: % defined as follows:
%
$$
\{\;  q(\vec x_{1}) \impl
\conj_{1}(\vec x_1,\vec y_1),\;
  \ldots,
  \;q(\vec x_m) \impl
\conj_{m}(\vec x_m,\vec y_m) \;\}
$$


% \[
%   \begin{array}{rrll}
%     \{&  q(\vec x_{1}) ~\impl~
%     &\conj_{1}(\vec x_1,\vec y_1),\\
%     &  \vdots&\\
%     &  q(\vec x_m) ~\impl~
%     &\conj_{m}(\vec x_m,\vec y_m) &\}\\
%   \end{array}
% \]
% 
We first replace each rule in $q$ with a variant thereof, so that in the end all
rules are standardized apart, as follows
%
$$
    \{\;  q(\vec x'_{1}) \impl
    \conj_{1}(\vec x'_1,\vec y'_1),
    \;  \ldots,\;
      q(\vec x'_m) \impl
    \conj_{m}(\vec x'_m,\vec y'_m) \;\}
$$
% \[
%   \begin{array}{rrll}
%     \{&  q(\vec x'_{1}) ~\impl~
%     &\conj_{1}(\vec x'_1,\vec y'_1),\\
%     &  \vdots&\\
%     &  q(\vec x'_m) ~\impl~
%     &\conj_{m}(\vec x'_m,\vec y'_m) &\}\\
%   \end{array}
% \]
where each primed variable is a fresh new variable. Clearly, the expression
above is equivalent to the original UCQ.

We then generate the following CQ:
$$
    q'(\vec x'_{1},\ldots,\vec x'_m) \impl
    \conj_{1}(\vec x'_1,\vec y'_1),
    \ldots,
    \conj_{m}(\vec x'_m,\vec y'_m)
$$
% \[
%   \begin{array}{rl}
%     q'(\vec x'_{1},\ldots,\vec x'_m) ~\impl~
%     \conj_{1}(\vec x'_1,\vec y'_1),
%     \ldots,
%     \conj_{m}(\vec x'_m,\vec y'_m)\\
%   \end{array}
% \]

Now we apply to $q'$ the same technique we have shown for CQs.
We obtain as output a Datalog program $\Pi$
whose first clause redefines $q'$ and has the following form:
%
\[
    \{q'(\vec x'_{1},\ldots,\vec x'_m) ~\impl~
    \retr{\conj}_{1}(\vec x'_1,\vec y'_1),
    \ldots,
    \retr{\conj}_{m}(\vec x'_m,\vec y'_m)\} \cup \Pi'
\]
where the various ${\retr{conj}_{i}}$'s are as the ${conj_{i}}$'s, but refer to
the cache relations instead, which are defined in the remainder of $\Pi$, which
we denote as $\Pi'$.

The above $m$ rules together with $\Pi'$ constitute a Datalog program
corresponding to an optimized query plan that computes all the answers to the
original UCQ $q$ that are obtainable from the relations in $\R$.

Finally, from the rewritten query $q'$, we generate the following $m$ rules.
%
$$
    \{\;  q'(\vec x'_{1}) \impl
    \retr{\conj}_{1}(\vec x'_1,\vec y'_1),
      \;\ldots,\;
      q'(\vec x'_m) \impl
    \retr{\conj}_{m}(\vec x'_m,\vec y'_m) \;\}
$$
% \[
%   \begin{array}{rrll}
%     \{&  q'(\vec x'_{1}) ~\impl~
%     &\retr{\conj}_{1}(\vec x'_1,\vec y'_1),\\
%     &  \vdots&\\
%     &  q'(\vec x'_m) ~\impl~
%     &\retr{\conj}_{m}(\vec x'_m,\vec y'_m) &\}\\
%   \end{array}
% \]
% TODO: adjust the conclusion of this.
The above $m$ rules together with $\Pi'$ constitute a Datalog program whose execution with the fast-failing strategy (with minor adjustments for UCQs) implements a 
%$\submin$-
minimal query plan for $q$.
Relevance can be easily decided for all predicates in $\R$ from the optimized d-graph of $q'$ in the same way as was done for CQs. 
% TODO: show how + theorem


\subsection{Adding safe negation}\label{sec:negation}

We now consider the class of \CQnot \ queries and show how to check relevance of relations.
Whenever $q$ is a \CQnot \ query, we indicate by $q^{+}$ (resp., $q^{-}$) the query whose head is the same as $q$'s and whose body is the conjunction of the negative (resp., positive) literals in $q$.
%
Clearly, if $q$ is unsatisfiable no relation is relevant;
else, any relation occurring in $q$ is necessarily relevant. A relation not occurring in $q$ is relevant if and only if it occurs in the optimized d-graph of $q^+$, which is a CQ. Indeed, safeness of negation implies that all variables occurring in a negative literal also occur in a positive one. In other words, a negative occurrence of a literal only needs those values that come from the positive literals in the body, which are considered in the optimized d-graph of $q^+$.
%
\begin{theorem}\label{the:dgraph-relevant-cqnot}
  Let $q$ be a \CQnot \ query over a schema $\R$ and let $G$ be the optimized d-graph for $q^+$.
  A relation $r\in\R$ is relevant iff $q$ is satisfiable, and
  \myi $r$ occurs in $q^{-}$, or
  \myii $r$ occurs in $G$, or
  \myiii $r$ is nullary and occurs in $q$.
\end{theorem}
%
Note that unsatisfiability for \CQnot \ queries can be checked in PTIME~\cite{LuNa04a} and the optimized d-graph of a CQ is also computed in PTIME;
this easily extends to \UCQnot \ queries.
%
\begin{corollary}
Relevance for \UCQnot \ queries is decidable in PTIME.
\end{corollary}
%
A %$\submin$-
minimal query plan for \UCQnot \ queries can be obtained similarly to what was done for UCQs.
%
We observe that a %$\submin$-
minimal execution strategy for a \CQnot query $q$ consists in: (i) executing the optimized query plan for $q^+$; (ii) filtering out the result set by calling the atoms in $q^-$ only with the bindings obtained from $q^+$, and never twice with the same bindings.
%
An alternative %$\submin$-
minimal execution strategy would consist in executing those parts of $q^+$ which can provide useful bindings for the atoms in $q^-$, and executing such atoms as soon as possible, possibly before other atoms in $q^+$.
%
Statistical information on the sources might suggest which strategy is more suitable, although neither will prevail on all instances.

\subsection{Datalog queries}\label{sec:datalog}

We have shown in the previous sections that it is possible to determine whether a given relation is relevant for a given query expressed as a \UCQnot.
We now prove that this is not the case for Datalog queries, since, if we were able to decide relevance of a relation, we could also decide containment between Datalog queries, which is known to be undecidable.
This solves negatively the issue opened in \cite{LiCh01b}.
% in 2001.

\begin{theorem}\label{the:relevant-datalog-undecidable}
There cannot be an algorithm that determines whether a relation is relevant for a Datalog query.
\end{theorem}
\begin{proof}(Sketch)
Let $\Pi$ be a Datalog program over a schema $\R$, without access limitations, defining two arbitrary predicates $p$ and $q$.
Let $e$ be an extensional predicate not occurring in $\Pi$ and $i$ a new intensional predicate defined by the rules 
$i(\vec x)\la e, p(\vec x)$, and $i(\vec x)\la q(\vec x)$.
%
If we could establish whether $e$ is relevant to answer the Datalog query
%
$Ans(\vec x) \la i(\vec x)$
%
then we could also decide containment between $p$ and $q$, which is absurd (and \emph{a fortiori} absurd for a schema that can have access limitations).
More precisely, $q$ contains $p$ iff $e$ is not relevant, i.e.:
\begin{enumerate}
\item[(1)] If $q$ contains $p$, then $e$ is not relevant.
\item[(2)] If $q$ does not contain $p$, then $e$ is relevant.
\end{enumerate}

To see that (1) holds, it suffices to observe that $r_1$ cannot contribute any tuple to $i$ that is not already contributed by $r_2$, and thus $e$ need not be accessed.

To see that (2) holds, consider a database $D$ in which $p(\vec c)$ holds but $q(\vec c)$ does not hold, for some constants $\vec c$. Such a database must exist, as $p$ is not contained in $q$.
Since $p$ and $q$ are independent of $e$, their containment is also independent of $e$, so $e$ may either hold or not hold in $D$. If $e$ holds in $D$, then $\vec c$ is in the answer; if $e$ does not hold in $D$, then $\vec c$ is not in the answer. Therefore $e$ is relevant, since it changes the obtainable answers for the query.
%\markempty
\end{proof}

% \begin{figure} 
%  \centering 
% \begin{tabular}{|r|c|c|c|c|c|}
% \hline
% &CQ&UCQ&\CQnot&\UCQnot&Datalog\\
% \hline
% Relevance&PTIME&PTIME&PTIME&PTIME&undecidable\\
% \hline
% $\exists$-Minimality&PTIME&PTIME&PTIME&PTIME&impossible?\\
% \hline
% $\forall$-Minimality&?&?&?&?&impossible?\\
% \hline
% $*$-Minimality&?&?&?&?&impossible?\\
% \hline
% Containment&$\in$coNEXPTIME&?&?&?&undecidable\\
% \hline
% \end{tabular}
%  \caption{Relevance, minimality, and containment}
%  \label{tab:relevance-minimality-containment} 
% \end{figure}
