\section{Discussion}\label{sec:discussion}

We have presented a novel technique to determine the relevant relations to a query formulated over relations with access limitations.
Based on the binding patterns of a query, we construct a d-graph that indicates all possible ways in which input arguments in a relation can receive useful values from output arguments with the same
domain. The d-graph is optimized according to the joins included in
the query and then used to generate a Datalog program that answers the
query. The technique works for CQs as well as UCQs.

\subsection{Conclusion from sigmod}
\label{sec:conclusion}

In this paper, we have presented a technique to determine sources that are
relevant to a certain query, expressed as a union of CQs with negation, under
access limitations; this solves a quite long-standing open problem, that was
left open for even positive CQs.  Deciding relevance can be done in deterministic
polynomial time.  Also, we have shown that thes same problem is undecidable for
Datalog queries.

We have started from a graph-based approach to represent how values extracted
from one source can serve to access another source; by suitably pruning this
graph, we are able to determine which sources are relevant.  Furthermore, we
have presented an algorithm to generate a query plan from the pruned graph;
such algorithm, evaluated according to a specific execution strategy, ensures
minimality of the number of accesses to sources, while returning all obtainable
answers.
% TODO: scrivere qualcosa su ICDE 2008
%  This is described in~\cite{CaMaICDE2008}, where the implementation of
% our system \system{} is presented.

Although further optimizations may be obtained that consider statistical
information on the sources (e.g., their expected response times), our
minimization is a necessary layer for efficient query answering, since all
accesses that are unnecessary for every instance are to be excluded anyway.

Finally, we have given experimental evidence of the effectiveness of our
technique to reduce the number of accesses.

Directions of future investigation that may have an immediate impact on query
optimization in this context include algorithms for efficiently checking query
containment under access limitations; known algorithms for query containment
that achieve the lowest computational complexity are now non-deterministic, and
it is not obvious how to derive from them practical algorithms, possibly
employing run-time optimization techniques.  Also, we plan to consider the
optimization of query answering where we have integrity constraints on top of
the access limitations.

% FIXME: qui o nelle conclusioni o dove?
% ANDREA: secondo me, meglio nelle conclusioni
We also observe that the techniques described in this paper continue to apply
even if additional information about useful binding values is known.  For
instance, to include data cached from previous query evaluations, it suffices
to include such caches in the source schema.  If, by domain knowledge or
statistical information, it is known that the sensible bindings for a given
domain are all within a given range or set, such set can, again, be modeled as
a free source; then, in the d-graph, it suffices to mark all its outgoing arcs
as candidate strong.

