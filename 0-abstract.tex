\begin{abstract}
	Access limitations are characterizations of a database schema that classify the attributes of a relation as either input or output.
	Such limitations occur, for instance, in data exchange and integration, data warehousing, and Web information systems, when heterogeneous sources, possibly accessible through web forms, are queried.
	Since, in such contexts, the full answer to a query cannot generally be computed in the same way as in a traditional database, one is typically interested in the \emph{maximally contained answer}, i.e., the set of all answer tuples that can be obtained in spite of the access limitations.
	Retrieving such answers requires, in general, a recursive evaluation process
%	  that may involve relations that are present in the schema, but that are not mentioned in the query.
	that involves all relations of the schema that are \emph{relevant} to the query, possibly including relations not mentioned in the query.

	In this paper, we develop a technique for determining whether a relation is relevant to a conjunctive query over a schema with access limitations. This solves a long-standing open issue.
	Furthermore, we complete the study of the problem by extending this technique to the context of unions of conjunctive queries with 
% TODO: vogliamo dire "safe"?
negation and by showing that relevance is undecidable for Datalog queries.
\end{abstract}
