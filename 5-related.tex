% VERSIONE SIGMOD 2008
\section{Related Work}
\label{sec:related}

Processing queries under access limitations is an important issue in database
theory and information systems.  Several works in the literature have
investigated this problem~\cite{RaSU95,DuLe97,FLMS99,MiLF00,LiCh00,LiCh01,LiCh01b,Li03,DBLP:journals/jcss/MillsteinHF03,NaLu04,LuNa04a,YaKC06,Srivastava:2006fk,Deutsch:2007lr}. In particular: \cite{FLMS99}
considers non-recursive plans and their optimization; \cite{RaSU95} addresses
the problem of determining whether a conjunctive query can be answered in the
presence of access limitations, showing that it can be decided in polynomial
time; finally, \cite{DuLe97} presents algorithms for query answering using
views under access limitations;
%
\cite{MiLF00,LiCh00} introduced recursive query
plans; \cite{MiLF00} addresses the problem of query containment
under access limitations.

% Since accessing data sources is costly, especially in the case of Web data, a
% fundamental issue is how to reduce the \emph{number} of accesses to the
% sources while still returning all obtainable answers to a query.

A few works address the problem of optimizations that can be made at the time
of the generation of the query plan~\cite{LiCh00,LiCh01,LiCh01b}.  All of these
works consider a narrow class of queries that they call \emph{connection
  queries}; such a class is a proper subclass of the class of UCQs, that does
not include CQs.  We do believe that connection queries are of little
significance w.r.t.~queries used in practice: indeed, in a connection query,
there has to be a join among all attributes that share the same abstract
domain; moreover, all such attributes have to be either all selected or all
non-selected.  If we consider a binary relation $\mathit{supervises}$ with both
attributes belonging to the abstract domain $\mathit{Employee}$, the
insufficient expressiveness of connection queries is evident: with such
queries, we can \emph{only} ask for those employees who are supervisors of
themselves (or, with a ground query, whether a given employee is supervisor of
him/herself); we cannot, for example, ask who is the supervisor of a certain
employee.  In our experimentations, we chose not to compare our optimizations
with those of~\cite{LiCh01b}, since most of the queries we dealt with are not
connection queries; notice that in case of connection queries we are able to
determine the relevant sources as in~\cite{LiCh01b}, and we are also able to
perform further optimizations w.r.t.~ the accesses.  Notice that
in~\cite{LiCh01b} it is conjectured that the technique used for connection
queries can be used, with slight variations, to determine relevance for CQs;
however, being the structure of connection queries much simpler than that of
CQs, it is evident that it is not possible to use the same optimization
technique also for CQs.

%%%%%%%%%%%%%%%%%%%%%%%%%%%%%%%%%%%%%%%%%%%%%%%%%%%%%%%%%%%%%%%%%%%%%%%%%%%%%


%%% ALTRI PROBLEMI %%%
Another issue in this framework is \emph{stability}, i.e., determining whether
the \emph{complete} answer to a query (the one that would be obtained with no
access limitations) can always be computed despite the access limitations; this
is addressed in~\cite{Li03}.
%
Checking the \emph{feasibility} of a query amounts to determining whether there
exists an equivalent query that is executable as is from left to right, while
respecting the access limitations.  This is studied
in~\cite{NaLu04,LuNa04a,YaKC06}.
% In particular, it is studied whether there is an ordering of subgoals that
% enables answering the query, and, if multiple such orderings are possible,
% how to pick the best ordering.
%
\cite{Deutsch:2007lr} addresses the (quite general) problem of query answering using
views~\cite{Hale01} under integrity constraints and under access limitations,
showing that access limitations can be encoded into integrity constraints of
suitable form, that can be treated together with the other constraints.  This
work, however, does not considers the optimization problem.

In \cite{Srivastava:2006fk}, the authors propose a framework for querying
multiple Web services. They describe an algorithm, based on statistical
information about Web services, for arranging a query's service invocations
into a pipelined execution plan that exploits parallelism among Web services so
as to minimize the query's total running time.  Their approach refers to a
context in which the queries to be handled are supposed to be always
feasible. In particular, cyclic input/output dependencies, such as those of
Example~\ref{exa:limitations} among $r_1$, $r_2$, and $r_3$, are not allowed;
in their simpler setting, query plans for CQs need no recursive evaluation.
However, in the general case, query plans may need a recursive evaluation even
when the query is conjunctive.
%
When values are available from multiple web services, it may be important to
decide which service is of higher quality.  This choice needs to be guided by
statistical considerations; profiling techniques enabling efficient query
processing in this context are described in \cite{SrivastavaPhDThesis}.
However, this issue is orthogonal to the notion of relevance studied in this
paper: irrelevant sources are those that are certainly not useful to answer a
query and can be ruled out independently of statistical information.

% TODO: aggiungere qualcosa riguardo ICDE 2008
% questo e' ICDE-08
% In~\cite{CaMaICDE2008}, where we first introduced the graph model and the algorithm
% for the generation of minimal query plans, we describe the system \system:
% \system{} deals with positive conjunctive queries only, and implements suitable
% heuristics to query sources in parallel as much as possible, thus reducing the
% overall query answering time.
% Notice also that \system{} accesses only relevant sources.

Access limitations have also been studied in the context of logic programs with
(input or output) \emph{modes}; in particular, the notion of
\emph{well-modedness}, due to \cite{DBLP:conf/slp/DembinskiM85}, corresponds to
the notion of executability of a query from left to right of the previously
mentioned works.
